\documentclass[uplatex, onecolumn, 10pt]{jsarticle}

\usepackage[dvipdfmx]{graphicx}
\usepackage{latexsym}
\usepackage{bmpsize}
\usepackage{url}
\usepackage{comment}

\def\Underline{\setbox0\hbox\bgroup\let\\\endUnderline}
\def\endUnderline{\vphantom{y}\egroup\smash{\underline{\box0}}\\}

\newcommand{\ttt}[1]{\texttt{#1}}

\begin{document}

\title{\vspace{-40mm}\bf{\LARGE{ゼミ報告書}}}
\author{\vspace{-40mm}木村 優哉 67200203}
\date{2023-05-19 Fri}
\maketitle


\section{今日までにやったこと}

\subsection*{研究関連}
\begin{itemize}
	\item OpenCV環境構築の続き
	\item OpenCVで表の位置情報を取得する機能実装
\end{itemize}

\subsection*{その他}
\begin{itemize}
	\item 説明会、面接参加
	\item ロボコンミーティング参加
	\item OpenCV, Python, Vueの勉強
\end{itemize}


\section{次までにやること}

\subsection*{研究関連}
\begin{itemize}
	\item 矩形領域取得の精度向上
	\item 二値化関数の引数を自動で決める
	\item 表の欄から属性を割り当てる方法を探す
\end{itemize}

\subsection*{その他}
\begin{itemize}
	\item 説明会・選考参加
	\item ロボコンミーティング参加
	\item OpenCV、Typescriptの勉強
\end{itemize}

\end{document}
