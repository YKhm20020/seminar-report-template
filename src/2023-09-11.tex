\documentclass[uplatex, onecolumn, 10pt]{jsarticle}

\usepackage[dvipdfmx]{graphicx}
\usepackage{latexsym}
\usepackage{bmpsize}
\usepackage{url}
\usepackage{comment}

\def\Underline{\setbox0\hbox\bgroup\let\\\endUnderline}
\def\endUnderline{\vphantom{y}\egroup\smash{\underline{\box0}}\\}

\newcommand{\ttt}[1]{\texttt{#1}}

\begin{document}

\title{\vspace{-40mm}\bf{\LARGE{ゼミ報告書}}}
\author{\vspace{-40mm}木村 優哉 67200203}
\date{2023-09-11 Mon}
\maketitle


\section{今日までにやったこと}

\subsection*{研究関連}
\begin{itemize}
	\item プロンプトの修正
	\item 実際に抽出した文字を対象に属性を判定
	\item 形態素解析で明確に誤検知である文字を除外
	\item MeCab から fugashi へ移行
	\item unidic-lite から unidic へ辞書を移行
\end{itemize}

\subsection*{その他}
\begin{itemize}
	\item ロボコンミーティング参加・実装
	\item 産学連携ミーティング
	\item C++の勉強
\end{itemize}


\section{次までにやること}

\subsection*{研究関連}
\begin{itemize}
	\item 実装した機能の連結
	\item 文字抽出および付加属性のデータ出力
	\item 誤検知除外機能の拡大
	\item DeblurGAN を適用
	\item Llama2 の論文を読む
\end{itemize}

\subsection*{その他}
\begin{itemize}
	\item ロボコンミーティング参加・実装
	\item 産学連携ミーティング
	\item C++ の勉強
\end{itemize}

\end{document}
