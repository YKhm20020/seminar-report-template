\documentclass[uplatex, onecolumn, 10pt]{jsarticle}

\usepackage[dvipdfmx]{graphicx}
\usepackage{latexsym}
\usepackage{bmpsize}
\usepackage{url}
\usepackage{comment}

\def\Underline{\setbox0\hbox\bgroup\let\\\endUnderline}
\def\endUnderline{\vphantom{y}\egroup\smash{\underline{\box0}}\\}

\newcommand{\ttt}[1]{\texttt{#1}}

\begin{document}

\title{\vspace{-40mm}\bf{\LARGE{ゼミ報告書}}}
\author{\vspace{-40mm}木村 優哉 67200203}
\date{2023-07-14 Fri}
\maketitle


\section{今日までにやったこと}

\subsection*{研究関連}
\begin{itemize}
	\item 下線部検出の手法を調査
	\item 余白のクリッピング機能実装
	\item 矩形領域ごとに画像を切り取り、文字認識する機能実装
\end{itemize}

\subsection*{その他}
\begin{itemize}
	\item ロボコンミーティング参加・実装
	\item 英論探し(Number Plate Detection Using YOLOV4 and Tesseract OCR)
	\item C++, Laravelの勉強
\end{itemize}


\section{次までにやること}

\subsection*{研究関連}
\begin{itemize}
	\item 下線部検出機能を実装
	\item 文字が誤って矩形領域として検出されないよう修正
\end{itemize}

\subsection*{その他}
\begin{itemize}
	\item ロボコンミーティング参加・実装
	\item 英論翻訳(A Novel Method for Image to Text Extraction Using Tesseract-OCR)
	\item C++, Laravelの勉強
\end{itemize}

\end{document}
